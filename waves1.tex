\chapter{Oscillazioni lineari adiabatiche. Modi di oscillazione.}
\PartialToc


\section{Per punti.}

\tool{
\begin{itemize}
    
    \item Perturbazioni: reazione del sistema: oscillazioni.
    
    \item Relazioni perturbazioni vs Langrangiana (Tolsoy): in entrambe introduco la perturbazione della posizione $\Lvar{\vec{\xi}}$, ma tolsoy gi\'a ignora la perturbazione di $g$.
    \item Adiabatic approximation: dalsgaard, stellar oscillation pg.47: why we can neglect heating term in energy equation.
    
    \item Nel caso di onde puramente acustiche
    \begin{align*}
    &\PtwoDy{t}{\rho'}=-v_S^2\nabla^2\rho'\intertext{equazione d'onda per la propagazione della perturbazione}\\
    &v_S=\sqrt{\frac{\Gamma_{1,0}P_0}{\rho_0}}&\intertext{adiabatic (Laplacian) sound speed.}
    \end{align*}

    Usando l'equazione di continuit\'a si vede che
    \begin{align*}
    &|\frac{\rho'}{\rho_0}|=\frac{v}{v_S}\\
    &|\frac{\rho'}{\rho_0}|\ll1\ \Rightarrow \ \frac{v}{v_S}\ll1
    \end{align*}
    La teoria lineare \'e valida finch\'e la velocit\'a delle fluttuazioni associata alle onde acustiche \'e minore rispetto alla velocit\'a del suono.
    Per l'equazione per la quantit\'a di moto linearizzata deve essere $\vec{c}\parallel \vec{k}$: la velocit\'a del fluido associato alle onde acustiche adiabatiche \'e parallela alla direzione di propagazione, sono ande di pressione.
    
    \item Forward problem: Matching accuracy of observations with accuracy of theoretical predictions.
    
    \item Metodo asintotico: varie approssimazioni. Accuratezza $10\%$. Le tecniche di inversione numerica hanno accuratezza \numrange{100}{1000} volte superiore ma dipende da un modello solare: dipendenza radiale dei coefficienti nelle equazioni delle oscillazioni.
    
    \item At observed solar frequencies the displacement at surface is approx. radial:
    
    \begin{equation*}
    \frac{\xi_h(R)}{\xi_r(R)}\approx\frac{GM}{R^3}\frac{L}{\omega^2}
    \end{equation*}
    
    \item L'approssimazione di Cowling \'e troppo grossolana se paragonata con l'accuratezza delle osservazioni \num{e-4}.
    (Vedi Robe68 JCD84)
    
    \item Confronto accuratezza asintotica vs numerica vs accuratezza osservazioni
    
    \item Procedure for determine n for computed modes of oscillation: Scufflaire 74, Osaki 75.
    
    \item Cowling approximation: System of equation of second order: Sturm-Liuville problem.
    \item Classification is invariant under continuus variation of equilibrium model: $\lambda=0$ Cowling approximation, $\lambda=1$ full case.
    
    \item Numerical inaccuracy: Van der raay, Palle roca cortes 1986.
    
    \item Mathematical classification often doesn't reflect the physical nature of the modes (Osaki75)
    
    \item Integrated energy: (relative) kinetic energy within a mode
    \begin{equation*}
    E_{n,l}=\frac{\int_0^R[\xi^2_r(r)+l(l+1)\xi_h^2(r)]\rho r^2\,dr}{4\pi M[\xi^2_r(R)+l(l+1)\xi_h^2(R)]}
    \end{equation*}
    
    \item dispersione energia del modo: effetti non adibatici (superficie), effetti non lineari (accoppiamento con altri modi, accoppiamento con flussi di materia)
    
    \item Trapping of modes.
    
    \item Reflection due to increase of $\omega_c$ in external layers before adiabatic approx break down: for what modes ??
    
    
    \item Identificazione dei modi: identificazione dei promontori nel diagramma \dgndi{} and of the lines in the power spectrum of the full disc oscillation signal.
    \item extrapolation to infinite number of grid points (shibahashi osaki 81)
    \item Pulsational unstable: self-excited oscillations.
    \item Eigenfrequencies for an infinite number of grid point is extrapolated $\nu_N=\nu_{\infty}+\frac{a}{N^2}$
    \item Forward problem: asymptotic expression for frequencies.
    \item dipendenza parametrica modello forward problem: asymptotic vs numerical.
\end{itemize}
}

In questa sezione descrivo le caratteristiche dei modi normali del Sole e come la struttura del interna del Sole influisce sulle frequenze.

Quando le frequenza sono molto grandi (per i modi p) o molto piccole (per i modi g) \'e possibile ricavare soluzioni analitiche approssimate delle equazioni delle oscillazioni. 

\section{Perturbazioni lineari adiabatiche.}

\subsection{Perturbatione dello stato di equilibrio.}

\begin{todo}{Cerca ampiezza media oscillazioni superficie ($\exv{v_{osc}}$)}
La piccola ampiezza delle oscillazioni giustifica l'uso solo del termine lineare dell'espansione.

\end{todo}

Descrivo le oscillazioni come piccole perturbazioni attorno allo stato di equilibrio stazionario (gli effetti non lineari sono dell'ordine di $\frac{v}{c_s}$ dove v \'e l'ampiezza dell'oscillazione):

\begin{align*}
&P(\vec{r},t)=P_0(\vec{r})+P'(\vec{r},t)&\intertext{$P'(\vec{r},t)$ \'e la perturbazione euleriana, quindi, detto $\delta\vec{\xi}$ lo spostamento della particella di fluido a causa della perturbazione}\\
&\Lvar{P(\vec{r})}=P(\vec{r}+\Lvar{\vec{\xi}})-P_0(\vec{r})=P'(\vec{r})+\Lvar{\vec{\xi}}\cdot\nabla P_0&\intertext{la velocit\'a dell'elemento di fluido dovuta alla perturbazione \'e}\\
&\vec{v}=\PDof{t}(\Lvar{\vec{\xi}})
\end{align*}

\begin{todo}{Particular solution/Perturbed solution}
Equations of conservation for stellar structure form a system of non-linear, partial differential equations. If an unperturbed solution is known we are often interested in finding another solution ''perturbed'' which differs only slightly from the unperturbed (we may think of the two solutions as representing possible future of the fluid differing from each others because of different initial conditions).

Expressing dependent variable of perturbed solutions as the sum of corr. dependent vars of unperturbed solution, neglecting all powers above the first and product of variations we obtain a system of partial differential equation whose solution gives the behaviour of the variation: the resulting set of equations is linear.

\end{todo}

Ricavo l'equazione del moto perturbato
\begin{align}
&\intertext{sostituisco nell'equazione del moto}
    &\rho\TDof{t}v\indices{_i}=\rho(\PDy{t}{v\indices{_i}}+v\indices{_j}\partial\indices{_j}v\indices{_i})=-\nabla P\indices{_i}+\rho\vec{g}\indices{_i}&\intertext{ le grandezze perturbate e sottraendo l'equazione statica ottengo}\nonumber\\
&\rho_0\PtwoDy{t}{\Lvar{\vec{\xi}}}=\rho_0\PDy{t}{\vec{v}}=-\nabla P'+\rho_0\vec{g}'+\rho'\vec{g}_0\label{eq:emper}\\
&\vec{g}'=-\nabla\Phi',\ \nabla^2\Phi'=4\pi G\rho'\nonumber
\end{align}

Analogamente per l'equazione di continuit\'a ottengo
\begin{equation}
\rho'+\div{(\rho_0\Lvar{\vec{\xi}})}=0\label{eq:contper}
\end{equation}

\subsection{Adiabatic approximation}

I tempi caratteristici per scambio di calore sono maggiori del periodo delle pulsazioni


\begin{align*}
&\TDy{t}{q}=\frac{1}{\rho(\Gamma_3-1)}(\TDy{t}{P}-\frac{\Gamma_1P}{\rho}\TDy{t}{\rho})=\epsilon-\frac{1}{\rho}\scap{\nabla}{F}&\intu{energy equation (rate heat gain/loss)}\\
&\frac{1}{\rho c_P}\nabla\cdot(\frac{4acT^3}{3\kappa\rho}\nabla T)\approx\frac{4acT^4}{3\kappa\rho^2c_PH}=\frac{T}{\tau_R}&\intertext{$\tau_R$ tempo scala radiativo, H lunghezza caratteristica, in cgs:}\\
&\tau_R=\num{e12}\frac{\kappa\rho^2H^2}{T^3}
\end{align*}

Per valori caratteristici solari ($\kappa=1$, $\rho=1$, $T=\num{e6}$, $H=\num{e10}$) ho $\tau_R\approx\SI{e7}{\year}\approx\tkh{}$, per valori caratteristici della zona convettiva ($\kappa=100$, $\rho=\num{e-5}$, $T=\num{e4}$, $H=\num{e9}$) ho $\tau_R\approx\SI{e3}{\year}\approx\tkh{}$.


In the inner part the nuclear term correspond to characteristic time $\tau_{\epsilon}\approx\frac{c_PT}{\epsilon}\approx\tkh{}$.

Confronto $\frac{T}{\tau_R}$, $\frac{T}{\tau_{\epsilon}}$ con $\TDy{t}{T}\approx\frac{T}{\Pi_{osc}}$ con $\Pi_{osc}\approx\si{\minute}-\si{\hour}$: heating term is generally very small compared with time derivative term.

Il moto di una elemento di fluido \'e descritto dalla relazione adiabatica


\begin{align*}
&\TDy{t}{P}=\frac{\Gamma_1P}{\rho}\TDy{t}{\rho}
\end{align*}

Approssimazione adiabatica non pi\'u valida vicino alla superficie solare dove i tempi per lo scambio di calore sono pi\'u brevi.

La condizione di perturbazione adiabatico linearizzata \'e
\begin{align}
&\PDy{t}{\Lvar{P}}-\frac{\Gamma_{1,0}P_0}{\rho_0}\PDy{t}{\Lvar{\rho}}=0\nonumber&\intertext{che integrata rispetto a t ed in funzione della variazione euleriana diventa}\nonumber\\
&P'+\Lvar{\vec{\xi}}\cdot\nabla P_0=\frac{\Gamma_{1,0}P_0}{\rho}(\rho'+\Lvar{\vec{\xi}}\cdot\nabla\rho_0)\label{eq:adper}
\end{align}

\subsection{Separazione variabili spaziali e temporali.}

Dall'equazione del moto \ref{eq:emper} si vede che
\begin{align*}
&\hat{r}\cdot(\rot{\PtwoDy{t}{\vec{\xi}}})=0&\intertext{cio\'e}\\
&\PDof{\theta}(\sin{\theta}\xi_{\phi})-\PDy{\phi}{\xi_{\theta}}=0&\intertext{quindi \'e possibile ricavare la componente tangenziale della perturbazione da una funzione scalare e dato che sono interessato alle oscillazioni }\\
&\vec{\xi}=\exp{i\omega t}(\xi_r(r),\xi_h(r)\PDof{\theta},\frac{\xi_h(r)}{\sin{\theta}}\PDof{\phi})Y_l^m(\theta,\phi)&\intertext{Ho introdotto le funzioni armoniche sferiche che soddisfano:}\\
&L^2Y_l^m=-\frac{1}{\sin{\theta}}\PDof{\theta}(\sin{\theta}\PDy{\theta}{Y_l^m})\\
&+\frac{1}{\sin^2{\theta}}\PtwoDy{\phi}{Y_l^m}=-r^2\nabla_h^2Y_l^m=l(l+1)Y_l^m
\end{align*}

La variazione euleriana di densit\'a, pressione, potenziale gravitazionale sono espressi
\begin{align*}
&(\rho_1,P_1,\Phi_1)=\exp{i\omega t}[\rho_1(r),P_1(r),\Phi_1(r)]Y_l^m
\end{align*}

\subsection{Frequenze di oscillazione discrete.}

Utilizzo l'equzione del moto ~\ref{eq:emper} e l'equazione di continui\'a~\ref{eq:contper} per eliminare $\xi_h(r)$ dall'equazione del moto
\begin{align}
&\frac{1}{r^2}\TDof{r}(r^2\xi_r)-\frac{\xi_rg}{c^2}+\frac{1}{\rho_0}(\frac{1}{c^2}-\frac{l(l+1)}{r^2\omega^2})P_1\nonumber\\
&-\frac{l(l+1)}{r^2\omega^2}\Phi_1=0\nonumber\\
&\frac{1}{\rho_0}(\TDof{r}+\frac{g}{c^2})P_1-(\omega^2-N^2)\xi_r+\TDy{r}{\Phi_1}=0\label{eq:eigenomega}\\
&\frac{1}{r^2}\TDof{r}(r^2\TDy{r}{\Phi_1})-\frac{l(l+1)}{r^2}\Phi_1-\frac{4\pi G\rho_0}{g}N^2\xi_r\nonumber\\
&-\frac{4\pi G}{c^2}P_1=0\nonumber
\end{align}

ho definito $N^2=g(\frac{1}{\Gamma_1P_0}\TDy{r}{P_0}-\frac{1}{\rho_0}\TDy{r}{\rho_0})$ e $S_l^2=\frac{l(l+1)c^2}{r^2}\approx k_h^2c^2$ con

\begin{align*}
&g=-\frac{1}{\rho_0}\TDy{r}{P_0}\\
&c^2=\frac{\Gamma_1P_0}{\rho_0}
\end{align*}


Il sistema di equazione ~\ref{eq:eigenomega} ha soluzione con le opportune equazioni al contorno per un insieme discreto di valori delle frequenze $\omega_{nlm}$, l'ordine angolare non compare nelle equazioni quindi gli autovalori $\omega_{nlm}$ sono $2l+1$ degeneri.

\subsection{Condizioni al contorno}

Abbiamo bisogno di 4 condizioni

\begin{itemize}
\item Due condizioni per $r=0$ punto regolare: le perturbazioni sono non singolari al centro del Sole, $r=0$.

\begin{equation*}
P'=0,\ \Phi'=0
\end{equation*}

Expansion near zero of solutions

\begin{align*}
&(l\neq0):\ \xi_r\propto r\expy{l-1};\ (l=0):\ \xi_r\propto r\\
&P',\ \Phi'\propto r^l
\end{align*}

\item Alla superficie solare richiediamo la continuit\'a di $\Lvar{\nabla\Phi}$ e che non si abbia dispersione verso l'esterno.

Outside the star $\rho'=0$ and Poisson equation can be solved by solution vanishing at infinity $\Phi'=Ar\expy{-l-1}$:
\begin{equation*}
\TDy{r}{\Phi'}+\frac{l+1}{r}\Phi'=0,\ r=\rsun{}    
\end{equation*}

The second condition depends on treatment of stellar atmosphere (Vedi chap 5 of lecture note on stellar oscillations: pg 103, (5.50)). It's reasonable that the boundary is free, no force acts on it: the star can be considered an isolated system. This is equivalent to requiring pressure constant at perturbed surface.

\begin{align*}
&\Lvar{P}=P'+\xi_r\TDy{r}{P}=0
\end{align*}

\end{itemize}

\subsection{Variabili adimensionali.}

Introduco le variabili adimensionali, che caratterizzano la perturbazione

\begin{align*}
&\eta_1=\frac{1}{r}\xi_r\\
&\eta_2=\frac{1}{gr}(\frac{P'}{\rho}+\Phi')\\
&\eta_3=\frac{1}{gr}\Phi'\\
&\eta_4=\frac{1}{g}\PDy{r}{\Phi'}\\
&\eta_i=\eta_i(r)Y_l^m(\theta,\phi)\exp{i\omega t}
\end{align*}

e riscrivo l'equazione del moto

\begin{equation*}
-\frac{\omega^2}{g}\vec{\xi}=[W(\eta_1-\eta_2+\eta_3)+(1-U)\eta_2]\hat{r}-r\nabla\eta_2
\end{equation*}

in funzione delle grandezze $U,V,W$ che caratterizzano lo stato di equilibrio del Sole

\begin{align*}
&U=\frac{r}{m}\PDy{r}{m}=\frac{1}{g}\PDy{r}{(gr)}\\
&V=-\frac{r}{P}\PDy{r}{P}=\frac{g\rho r}{P}\\
&W=\frac{r}{\rho}\PDy{r}{\rho}-\frac{r}{P\gamma_{Ad}}\PDy{r}{P}
\end{align*}

posto $\gamma_{ad}=\Dcvar{\TDly{\rho}{P}}{ad}=\Gamma_1$

La parte tangenziale dell'equazione del moto
\begin{align*}
&\frac{\omega^2}{g}\xi_{\theta}=\PDy{\theta}{\eta_2},\ &\frac{\omega^2}{g}\xi_{\phi}=\frac{1}{\sin{\theta}}\PDy{\phi}{\eta_2}
\end{align*}
sostituita nell'equazione di continuit\'a ($\scap{\nabla}{\xi}$), definita la frequenza adimensionale 
\begin{equation*}
\frac{\omega^2r}{g}=C\sigma^2:\ \sigma^2=\omega^2\frac{R^3}{GM}
\end{equation*}

permette di eliminare la dipendenza dalle variabili angolari
\begin{align*}
&r\PDy{r}{\eta_1(r)}=(3-\frac{V}{\gamma_{Ad}})\eta_1(r)+[\frac{l(l+1)}{C\sigma^2}\\
&+\frac{V}{\gamma_{Ad}}]\eta_2(r)-\frac{V}{\gamma_{Ad}}\eta_3
\end{align*}

mentre la parte radiale dell'equazione del moto

\begin{equation*}
r\PDy{r}{\eta_2(r)}=(W+C\sigma^2)\eta_1(r)+(1-U-W)\eta_2(r)+W\eta_3
\end{equation*}

dalla definizione di $\eta_3$

\begin{equation*}
r\PDy{r}{\eta_3}=(1-U)\eta_3(r)+\eta_4
\end{equation*}

infine l'equazione di Poisson \'e equivalente a
\begin{equation*}
r\PDy{r}{\eta_4}=-UW\eta_1+\frac{UV}{\gamma_{Ad}}\eta_2+[l(l+1)-\frac{UV}{\gamma_{Ad}}]\eta_3-U\eta_4
\end{equation*}



Abbiamo ottenuto quattro equazioni differenziali a coefficienti reali che dipendono dallo stato di equilibrio del modello stellare per le variabili adimensionali $\eta_i(r)$: un problema agli autovalori per $\sigma^2$, si pu\'o vedere che \'e autoaggiunto e quindi le autofunzioni corrispondenti ad autovalori diversi sono ortogonali: gli autovalori sono reali quindi posso avere un comportamento oscillante nel caso di stabilit\'a o esponenziale nel caso instabile.

Il sistema non dipende da m: le soluzioni sono $(2l+1)$ volte degeneri: la degenerazione \'e rimossa dalla rotazione ($\frac{\Omega}{\omega}\approx\num{e-4}$) o effetti gravitazionali di altri corpi.


\section{Stabilit\'a dei modi di oscillazione.}

\begin{todo}{Stabilit\'a oscillazioni nonradiali adiabatiche}

\end{todo}
