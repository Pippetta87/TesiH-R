\documentclass[../main.tex]{subfiles}

\begin{document}


\chapter{Introduzione al modello solare.(alle oscillazioni solari.)}
\PartialToc


\section{Eliosismologia.}

\subsection{Per punti}
\begin{itemize*}
\item Onde, modi normali
\item Cos'e l'eliosismologia.
\item info about theSun from surface oscillation.
\item Model test against observations.
\item Per molti modi il limite superiore non \'e lontano dalla fotosfera: tunnel attraverso la regione evanescente nell'atmosfera visibile.
\item principali risultati:
    \item Measurement of frequencies determine value of X,Y in outer envelope.
\item Come \'e divisa la tesi.



\item Doppler Measurement intensity and oscillation in atmospheric T.

Le oscillazioni interne si manifestano come moti oscillatori in atmosfera, cambiamenti nella potenza irradiata dal sole e campiamenti delle propriet\'a delle linee spettrali causati da oscillazioni di temperatura nell'atmosfera.
\item Evoluzione strumentazione. Doppler compensator. Resonant scatteing. Spectroheliograph. Pyrheliometer.
\item Scattering cell (Fossat,ricort 75; Brookes 76; Cacciani Fofi78; Rhodes 86; Tomczyk 95)
\item Fourier tachometer (Brown 84)
\item Broadband intensity observation from space (Woodard hudson 83; toutain frohlich 92)
\item Heliosismic facilities (Duvall 95)
\item Network observing stations: Bison (81): resont scattering cell, disk averaged velocity obs., (chaplin 96); GONG (95): Fourier tachometer (harvey 96), spatially resolved velocity observations; LOWL(Tomczyk95): magnetoptical filter; IRIS (Fossat 91); TON (Chou 95).
\item Space observations. SOHO(Domingo95). GOLF: Detection low frq. modes (may be g) integrated on whole disk, resonant scattering doppler velocity sodium cell (Gabriel 95-97). VIRGO (Variability of solar irradiance and gravity oscillations): g modes in intensity data (Frohlich 95-97). SOI/MDI: Fourier tachometer, Spatial resolution of \SI{2}{\arcsec}, oscillation up to degree l=1000 (4000??)
\item  Asteroseismology and Interferometry (0709.4613)
\item Leighton62:
\item Fondere parte meccanica della sezione leggi di conservazione fino all'equazione delle perturbazioni linearizzate e periodo fondamentale.
\item Struttura di equilibrio ($\TtwoDy{t}{r}=0$). Pulaszioni: variazione equazione del moto. (Vedi collins 3.1 pg 48)


\item Quali grandezze della struttuara solare hanno effetto diretto sulle oscillazioni.

Le frequenze delle oscillazioni adiabatiche sono determinate dal profilo radiale $P,\ \rho,\ ,g,\ \Gamma_1$: assumendo valide le relazioni basilari della struttura stellare  sono indipendenti 2  quantit\'a ($\rho(r),\Gamma_1(r)$ per esempio); viceversa le frequenze forniscono informazioni derette solo su queste quantit\'a (meccaniche).

\item At observed solar frequencies the displacement at surface is approx. radial:
\begin{equation*}
\frac{\xi_h(R)}{\xi_r(R)}\approx\frac{GM}{R^3}\frac{L}{\omega^2}
\end{equation*}

\item Deubner75: power ridge in \dgndi{} with slit
\item Ridge not resolved into single mode frequency ($\Delta l=1$: Solar circumference scan length??): spherical harminics are orthogonal on the full sphere. Single frequncy value can be computed by approximating power at each l by a smooth curve and finding the max of this curve $P(l,\nu)$.
\item Stellar pulsation modulate the emergent flux by their influence on the location of, and the physical conditions at the surface $\tau=\frac{2}{3}$.
\item strumenti di misura: resonance-scattering spectrometer
\item Claverie79: integrated light
\item Quality factor $Q=\frac{\nu}{\Delta\nu}$($\approx\numrange{e3}{e4}$: weak damping)
\item evoluzione doppler measurement
\item Trasformata di fourier/frequenza di Nyquist
\item Prodotto $G\msun$
\item Incertezza su G
\item incertezze parametri standard
\end{itemize*}


L'osservazione di fenomeni periodici nelle stelle permette di dedurre informazione sulla struttura.

Le stelle nella fase di MS  sono caratterizzate da numerosi modi di oscillazione di piccola ampiezza: lo studio delle osciallazioni della superficie solare e l'estrapolazione delle informazioni sulla struttura interna in essi contenuta \'e detta eliosismologia. \'E possibile calcolare numericamente le frequenze sulla base di un modello stellare e al variare di uno o pi\'u parametri del modello analizzare la corrispondenza con quelle osservate inoltre sono state sviluppate tecniche di inversione per dedurre la struttura e la dinamica interna del sole dalla misura delle frequenze.

\subsection{Equilibrio idrostatico.}

Le frequenze sono determinate principalmente dalla stratificazione (e dinamica) della regione in cui le ampiezze sono apprezzabili. 

\subsubsection{Distrubuzione della massa.}

Considero una distribuzione di massa sferica.

La massa presente in un guscio infinitesimo 
\begin{align*}
&dm=4\pi r^2\,dr-4\pi r^2\rho v\,dt&\intertext{equivalente ad all'equazione di continuit\'a}\\
&\PDy{t}{\rho}+\nabla\cdot(\rho\vec{v})=0
\end{align*}

La forza totale agente su un volume V di superficie S \'e
\begin{equation*}
\int_V\vec{F}\rho\,dV+\int_S\vec{t}(\vec{n},\vec{x},t)\,dS
\end{equation*}
dove $\hat{n}$ \'e la normale in ciascun punto di S, e $\vec{F}$ \'e una possibile body force per unit mass.

Lo stress su una superficie con normale parallela ad un asse coordinato ($\hat{n}=\hat{e}_i$ \'e
\begin{equation*}
\vec{\sigma}_i=\vec{t}(\hat{e}_i,\vec{x},t)=(\sigma_{i1},\sigma_{i2},\sigma_{i3}
\end{equation*}

quindi si ha che gli elementi $\sigma_{ij}(\vec{x},t)$ formano un tensore.

La conservazione del momento richiede
\begin{align*}
&\rho\TDy{t}{\vec{v}}=-\nabla\tensor{P}{_\cdot_\cdot}+\rho\vec{f}\\
&\intertext{per pressione termodinamica cio\'e trascurando viscosit\'a molecolare e radiativa, campi magnetici e turbolenze}\\
&\rho\TDy{t}{\vec{v}}=-\nabla P+\rho\vec{f}
\end{align*}
Ottengo quindi la condizione di equilibrio idrostatico $\ddvec{r}=0$:
\begin{equation}
\nabla P=\rho f\label{eq:idrosta}
\end{equation}


\subsubsection{Potenziale gravitazionale.}

Esplicito la forma della forza per unit mass f:
\begin{equation}\label{eq:gravitya}
g=\frac{Gm(r)}{r^2}
\end{equation}

Il potenziale gravitazionale \'e soluzione dell'equazione di Poisson 
\begin{align}
&\nabla^2\Phi=4\pi G\rho\label{eq:poisson}\\
&g=\PDy{r}{\Phi}=\frac{Gm(r)}{r^2}
\end{align}

Sostituendo nell'equazion di equilibrio idrostatico
\begin{equation}
\TDy{r}{P}=-\frac{Gm(r)\rho(r)}{r^2}\label{eq:idrostae}
\end{equation}


\subsubsection{Tempo di evoluzione dinamico.}

Scrivo l'equazione del moto per unit\'a di superficie di un guscio sferico
\begin{align*}
&\frac{dm}{4\pi r^2}\PtwoDy{t}{r}=f_P+f_g&\intertext{f \'e una forza per unit\'a di superficie,}\\
&\frac{1}{4\pi r^2}\PtwoDy{t}{r}=-\PDy{m}{P}-\frac{Gm}{4\pi r^4}
\end{align*}
Il valore tipico della derivata \'e approssimato dal rapport delle quantit\'a
\begin{align*}
&\tau_{ff}\approx\sqrt{\frac{R}{g}}\\
&\tau_{esp}\approx R\sqrt{\frac{\rho}{P}}
\end{align*}

Nelle stelle in cui l'equilibrio idrostatico \'e una buona approssimazione il tempo caratteristico di reazione a perturbazione dell'equilibrio \'e

\begin{align*}
&\tau_{idro}\approx \sqrt{\frac{R^3}{GM}}\approx\frac{1}{2}(G\overline{\rho})\expy{-\frac{1}{2}}\\
&G\msun=\num{1.32712440018e20}\pm\num{8e9}\si{\cubic\meter\per\square\second}\\
&\tau_{idro}^{\odot}\approx\SI{27}{\minute}
\end{align*}

\subsection{Teorema del viriale: equilibrio idrostatico.}

\begin{align*}
&\frac{1}{2}\TtwoDy{t}{I}=2K+\Omega\\
&2K=\sum_im_iv_i^2=\sum_i\scap{p_i}{v_i}&\intertext{$\scap{p_i}{v_i}$ measure rate of momentum transfer hence must be related to Pressure}\\
&P=\frac{1}{3}\int_pn(\vec{p})\scap{p}{v}d^3p&\intertext{confrontando le ultime due equazioni si ha:}\\
&2K=3\int_VP\,dV=3\int_M\frac{P}{\rho}\,dm(r)
\end{align*}

Il teorema del viriale si riscrive

\begin{align*}
&\frac{1}{2}\TtwoDy{t}{I}=\int_M\frac{3P}{\rho}\,dm(r)+\Omega&\intertext{Per equilibrio idrostatico}\\
&\frac{1}{2}\TtwoDy{t}{I}=0
\end{align*}

\subsection{\texorpdfstring{$\gamma$-law }{gamma-law} equation of state.}

Se vale una relazione del tipo $P=(\gamma-1)\rho u$ (per i gas ideali monoatomici con $\gamma=\frac{5}{3}$, $P=\frac{2}{3}\rho u$)
\begin{equation*}
2K=3(\gamma-1)\int u\,dm
\end{equation*}

$K=E_i$ solo per $\gamma=\frac{5}{3}$: l'energia cinetica \'e uguale all'energia interna totale solo in determinate circostanze.

Il teorema del viriale si riscrive
\begin{equation*}
3(\gamma-1)E_i+\Omega=0
\end{equation*}

e scrivendo l'energia totale $W=E_i+\Omega$ ottengo la relazione esplicita tra energia totale ed energia potenziale gravitazionale per stelle idrostatiche in cui vale la relazione $P=(\gamma-1)\rho u$
\begin{equation*}
W=\frac{3\gamma-4}{3(\gamma-1)}\Omega
\end{equation*}

\subsection{Teorema del viriale.}

Il teorema del viriale esprime una relazione statistica tra particelle interagenti: in particolare ricavo una relazione tra energia interna e energia potenziale gravitazionale.

L'energia potenziale gravitazionale della stella
\begin{equation}
\Omega=-\int_0^M\frac{Gm(r)}{r}\,dm\label{eq:energiapg}
\end{equation}

L'energia interna per unit\'a di massa per un gas ideale (monoatomico) u si esprime
\begin{equation}
\frac{P}{\rho}=\frac{R}{\mu}T=(\gamma-1)c_vT=\frac{2}{3}u\label{eq:energiaigp}
\end{equation}
Il teorema del viriale stabilisce che, posto $E_i=\int_0^Mu\,dm$,
\begin{equation}
E_g=-2E_i\label{eq:virialegpm}
\end{equation}

Per un'equazione di stato generale definisco il parametro $\zeta$
\begin{equation}
\zeta u=3\frac{P}{\rho}
\end{equation}
Per un gas ideale $\zeta=3(\gamma-1)\xrightarrow{\gamma=\frac{5}{3}}2$.

Per un gas di fotoni
\begin{align}
&P=\frac{1}{3}aT^4\label{eq:pressurephg}\\
&u\rho=aT^4,\ \zeta=1
\end{align}

Per $\zeta$ costante nella stella il teorema del viriale prende la forma

\begin{equation}
\zeta E_i+E_g=0\label{eq:virialezetac}
\end{equation}


\subsection{Particelle interagenti tramite potenziale funzione omogenea di grado n delle coordinate}

\begin{align*}
&2K-nU-3PV=0\\
&E=U+K&\intu{energia totale (energia interna?)}\\
&(n+2)K=nE+3PV
\end{align*}


\subsection{Modo fondamentale di oscillazione}

Le oscillazioni solari sono in prevalenza acustiche, legate al gradiente della pressione, e quindi determinate dal profilo radiale della velocit\'a del suono.


Per un corpo in equilibrio idrostatico ricavo il valore medio della velocit\'a del suono utilizzando il teorema del viriale
\begin{align*}
    &-\Omega=3\int_VP\,dV=3\int_M\frac{P}{\rho}\,dm=3\int_M\frac{v_s^2}{\Gamma_1}\,dm\\
    &=3\exv{\frac{v_s^2}{\Gamma_1}}M\approx3\frac{\overline{v}_s^2}{\gamma_1}M
\end{align*}

Se scrivo $\Omega=q\frac{GM^2}{R}$, per stelle di sequenza principale ho che $q\approx1.5$.

e quindi per il modo fondamentale di oscillazione radiale
\begin{align*}
    &\lambda_1\approx 2\rsun{}\\
    &\omega_1\approx\frac{c}{\lambda_1}\approx\SI{1}{\hour}
\end{align*}

\begin{todo}{considerazioni su $\Gamma_1$}
periodo fondamentale tenendo conto di $\Gamma_1$ 
\end{todo}

\subsection{Oscillazioni dei 5 minuti.}

Leighton62 osserva che la superficie solare ha scale spazio-temporali privilegiate: in particolare \'e presente un comportamento periodico nell'atmosfera a tutte le altezze rilevato tramite effetto doppler. Il periodo \'e di circa 300 secondi.

Il modello proposto da Ulrich70 e stein leibacher 71 considera le propriet\'a delle perturbazioni all'interno del Sole, in particolare dalla relazione di dispersione per onde acustiche si ha la definizione di cavit\'a risonanti al di sotto della superficie solare: sono possibili onde stazionarie per determinati valori di  $(k_h,\omega)$, dove $k_h$ \'e il numero d'onda orizzontale.

\subsection{Modi di oscillazione (Onde Stazionarie). Cavit\'a risonanti.}

I modi osservati hanno $\nu\geq\SI{500}{\micro\hertz}$: sono modi p (onde stazionarie: oscillazioni velocit\'a temperatura sfasati di \ang{90}) e modi f di alto grado (onde di gravita di una superficie libera).

Le vibrazioni libere di un corpo finito o comunque con condizioni ai bordi sono onde stazionarie la cui parte reale \'e del tipo $f(x,y,z)\cos{(\omega t+\alpha)}$: in assenza di effetti dissipativi la velocit\'a di fase \'e nulla e la velocit\'a di gruppo infinita.


Un'onda stazionaria in direzione radiale implica che  l'integrale di $k_r$ nella regione di propagazione sia un intero multiplo di $\pi$

\begin{align}
&(n+\alpha)\pi\approx\int_{r_t}^Rk_r\,dr\approx\int_{r_t}^R\frac{\omega}{c}\sqrt{1-\frac{S_l^2}{\omega^2}}\,dr&\intertext{ho usato la relazione di dispersione per onde acustiche e la frequenza di Lamb $S_l$}\\
&\omega^2=c^2|\vec{k}|^2,\ S_l^2=\frac{l(l+1)c^2}{r^2}
\end{align}

quindi il perido di un modo con $k_h$ fissato \'e determinato da

\begin{align}
    &(n+\frac{1}{2})\pi=\int k_r\,dr=\omega\int\frac{dr}{c}=\frac{2\omega^2}{(\gamma-1)gk_h}&\intertext{quindi si ha una curva parabolica compatibile con quelle osservate:}\\
    &\omega_n^2=\frac{(n+\frac{1}{2})\pi(\gamma-1)gk_h}{2}\approx(n+\frac{1}{2})gk_h&\intertext{dove l \'e il numero di lunghezze d'onda in una circonferenza solare:}\\
    &\lambda_h=\frac{2\pi}{k_h}\approx\frac{2\pi R}{\sqrt{l(l+1)}}
\end{align}


\begin{usefull}{Stima profondit\'a cavit\'a acustica}

La profondit\'a della cavit\'a acustica varia con il variare della scala orizzontale dell'onda: considero una stratificazione adiabatica

\begin{align*}
    &T=\Dcvar{\TDy{z}{T}}{Ad}\delta&\intertext{$\delta$ \'e la profondit\'a sotto la fotosfera}\\
    &\Dcvar{\TDy{z}{T}}{Ad}=\frac{T}{P}\TDly{P}{T}|_{Ad}\TDy{z}{P}=\frac{\Gamma_2-1}{\Gamma_2}\frac{\mu}{R}g=\frac{g}{c_P}&\intertext{$c_P$ \'e il calore specifico a pressione costante per unit\'a di massa. Scrivendo $c^2=(\Gamma_3-1)g\delta$, da $c=\frac{\omega}{k_h}$ al raggio per cui $k_r=0$ segue:}\\
    &\delta=\frac{\omega^2}{k_h^2(\Gamma_3-1)g}
\end{align*}

I modi con stesso $\frac{\omega}{k_h}$ sono confinati nella stessa cavit\'a.

\end{usefull}

L'analisi tramite FFT (della frequnza e del numero d'onda) delle osservazioni della superficie solare di deubner75 confermano che la  potenza delle oscillazioni (con k piccolo: $k=\frac{2\pi}{\lambda}<\SI{1}{\per\mega\meter}$) si distribuisce in linee determinate nel diagramma $(k_h,\omega)$ predette dal modello, mostra che sono provocate da modi acustici non radiali degli strati interni alla fotosfera: la concentrazione della potenza a bassi numeri d'onda indica che siamo in presenza di un fenomeno globale.

In particolare vengono effettuate delle scansioni lineari per \SI{300}{\arcsec} sulla superficie solare ogni \SI{110}{\second} con un'apertura di $2.0\times2.5$ \si{\arcsec}, e tramite lo shift Doppler della line del CI $5380$ viene misurata la velocit\'a lungo la linea di vista.

La larghezza dello spettro risonante ($Q=\frac{\Delta\nu}{\nu}$ \'e il rapporto tempo di crescita(o tempo di smorzamento)/periodo) riflette la legge di dispersione e la rapidit\'a di crescita/dissipazione di alcuni modi nella bassa fotosfera piuttosto che casualit\'a del processo.



Claverie 1979/80 osserva nello spettro Doppler (Neutral K line: \SI{769.9}{\nano\meter}) della luce integrata sull'intero disco solare delle frequenze equispaziate circa \SI{68}{\micro\hertz} interpretate come modi p di alto ordine l e basso grado l.

\begin{todo}{A resonant scattering spectrometer}

\end{todo}

\begin{todo}{Integrated sunlight}
grec83 moltitude and sharpness of the line in the solar low-l oscillation spectrum
\end{todo}

\subsection{Analisi modale.}

Osservando il campo di velocit\'a $v(x,y,t)$ sulla superficie solare ottengo la distribuzione della potenza delle oscillazioni $P(k_x,k_y,\omega)$

\begin{align}
    &v(x,y,t)=\int f(k_x,k_y,\omega)\exp{i(k_xx+k_yy+k_zz+\omega t)}\,dk_x\,dk_y\,d\omega\\
    &P(k_x,k_y,\omega)=ff^*\\
    &P(k_h,\omega)=\frac{1}{2\pi}\int_0^{2\pi}P(k_h\cos{\phi},k_h\sin{\phi},\omega)\,d\phi&\intertext{non esiste direzione privilegiata sulla superficie: la dipendenza \'e solo da $k_h=\sqrt{k_x^2+k_y^2}$.}
\end{align}

Un segnale di durata T permette una risoluzione $\Delta\omega=\frac{2\pi}{T}$: se devo risolvere due frequenze $\omega$ e $\omega+\Delta\omega$ devo osservare per un tempo $T=\frac{2\pi}{\Delta\omega}$ la frequenza pi\'u bassa osservabile \'e $\Delta\omega$, il limite superiore delle frequenze osservate \'e dato dalla risoluzione temporale $\Delta t$, la frequenza di Nyquist $\omega_{Ny}=\frac{\pi}{\Delta t}$ e analogamente per le variabili spaziali e numero d'onda associato
\begin{align}
&\Delta\omega=\frac{2\pi}{T}\leq\omega\leq\frac{\pi}{\Delta t}\\
&\Delta k_x=\frac{2\pi}{L_x}\leq k_x\leq\frac{\pi}{\Delta x}
\end{align}


Quando la dimensione dell'area osservata \'e comparabile con il disco solare tengo conto della geometria sferica
\begin{align}
    &v(\theta,\phi,t)=\sum_{l=0}^{\infty}\sum_{m=-l}^la_{lm}(t)Y_{lm}(\theta,\phi)\\
    &P(l,\nu=\frac{\omega}{2\pi})=a(\omega)a(\omega)*&\intertext{$a(\omega)$ \'e la trasformata di Fourier di $a_{l0}(t)$.}
\end{align}

Pi\'u precisamente il segnale \'e proporzionale alla velocit\'a proiettata lungo la linea di vista. Per modi con l basso o intermedio le oscillazione sono in direzione radiale. Prendendo l'asse delle armoniche sferiche sul piano del cielo ortogonale alla linea di vista, il segnale Doppler osservato \'e

\begin{equation}
    V_D(\theta,\phi,t)=\sin{\theta}\cos{\phi}\sum_{n,l,m}A_{nlm}c_{lm}P_l^m(\cos{\theta})\cos{(m\phi-\omega_{nlm}t-\beta_{nlm})}
\end{equation}
il fattore $\sin{\theta}\cos{\phi}$ deriva dalla proiezione della velocit\'a radiale sulla linea di vista.

Per isolare il contributo di una singola $Y_{l_0m_0}$ considero
\begin{align}
    &V_{l_0m_0}(t)=\int_AV_D(\theta,\phi,t)W_{l_0m_0}(\theta,\phi)\,dA\\
    &=\sum_{n,l,m}S_{l_0m_0,lm}A_{nlm}\cos{(\omega_{nlm}t+\beta_{nlm,L_0m_0})}\\
    &S_{l_0m_0,lm}\propto\delta_{ll_0}\delta_{mm_0}&\intu{funzione di risposta,}\\
    &W_{l_0m_0}\approx Y_{l_0m_0}
\end{align}

In pratica $V_{l_0m_0}(t)$ contiena contributi da valori di $(l,m)$ vicini.

La trasformata di Fourier di $V_{l_0m_0}(t)$ permette di isolare i signoli modi caratterizzati dall'ordine radiale n.


In linea di principio:
\begin{itemize}
    \item Dall'andamento di un modo sulla superficie solare si ricava $(l,m)$.
    \item L'ordine radiale n si ricava dalla distribuzione delle frequenze di oscillazione.
\end{itemize}


\subsection[???]{Risoluzione delle osservazioni.}


Large telescopes, dedicated to long seismic observations. Stable sensitive detector.

Doppler velocities/intensity fluctuations: p modes, Stable doppler image of entire disk; g modes, separate from noise of earth atmosphere and solar convective motion.

Due modi separati di $\Delta\nu$ sono risolti con osservazione di $T\geq\frac{1.5}{\Delta\nu}$ (T(hour)=417/($\Delta\nu(\si{\micro\hertz})$))

\begin{itemize}
    \item even/odd l: $T\geq \SI{6}{\hour}$
    \item modi l: $T\geq \SI{40}{\hour}$
    \item Rotational splitting: $T\geq \SI{400}{\hour}$
\end{itemize}

L'atmosfera solare \'e sudivisa in photosfera circa \SI{100}{\kilo\meter} e cromosfera, la parte pi\'u esterna: nella fotosfera il gas cambia da quasi trasparente a completamente opaco. La luce che riceviampo dal Sole \'e emessa dalla fotosfera.

From space helio (Toutain):

\begin{itemize}
    \item Ground network: Bison, Iris, Gong, Ton.
    \item Soho (Space)
\end{itemize}

Fino anni '80:

\begin{itemize}
    \item Ground: interruption N/D, Whether cond.
    \item first round clock observation grec81.
    \item filling method grec80.
    \item Osservazioni ininterrotte aumentano la risoluzione in $\omega$ e il rapporto S/N per modi con tempo di vita maggiore del tempo di osservazione.
    \item Rotational splitting $\Delta\nu=0.45\si{\micro\hertz}$.
    \item lines of modes below \SI{2}{\milli\hertz} have lifetime approx 1 month.
    \item Il rumore solare aumenta con la frequenza.
    \item Effetto dell'atmosfera: osservazioni di basso l hanno rumore a basse frequenze.
    \item Alto l sono affette da perdita di coerenza sul disco solare: leakage of high degree modes ($l>300$ sono coperte gi\'a da seeing di \ang{;;4}).
\end{itemize}

--(Space)--

ACRIM (active cavity radiometer irradiance monitor).

\begin{itemize}
    \item Orbital period \SI{95}{\minute} (\SI{35}{\minute} notte ??): Side band at \SI{+-170}{\micro\hertz}.
    \item Total irradiance measure: accuratezza maggiore di $0.1\%$ (possibile identificare i modi p in luce integrata).
    \item Osservazioni di 10 mesi: spettro modi p (woodard84) $l=0,1,2$, in un range di frequenze $\nu=\numrange{2.5}{3.8}\si{\milli\hertz}$.
    \item shutter cycle 131 secondi: occuratenza nelle frequenze di \SI{0.4}{\micro\hertz}.
    \item S/N da 1-4.
\end{itemize}

IPHIR (PhobosII).

\begin{itemize}
    \item Misure di luce integrata a 3 lunghezze d'onda: 3 interference filter \SIlist{335;500;862}{\nano\meter}.
    \item Accuratezza circa 1ppm.
    \item S/N circa 20 per modi p a \SI{3}{\milli\hertz}.
    \item p-modes ($l=0,1,2$): $\nu=\SIrange{2.4}{3.8}{\milli\hertz}$.
    \item Amplitude changes strongly with time (p-mode are stochastically excited by turbolent convection).
\end{itemize}

Virgo(SOHO: L1 Sole-Terra).

\begin{itemize}
    \item modi p di basso grado 
    \item Spectral and total irradiance: radiometer, fotometro con filtri a interferenza, Si-diode detector: $l\leq3$.
    \item Loi: $l\leq7$.
\end{itemize}

Golf(SOHO)
\begin{itemize}
    \item Misura spostamento Doppler di luce integrata sul disco solare: vapori di sodio (linee di Na: D1, D2).
    \item Modi p e g di basso grado angolare.
\end{itemize}

SOI/MDI(SOHO)
\begin{itemize}
    \item MDI (Michelson doppler image): fourier tachometer tuned across Ni-line (\SI{676.8}{\nano\meter}).
    \item Modi p con grado angolare medio-alto $L\leq4000$.
    \item Correlazione dei segnali velocit\'a/intensit\'a: effetti non adiabatici nella fotosfera.
\end{itemize}


\section{Onde}

\subsection{Onde EM}

Equazione di Laplace in coordinate sferiche: Jackson pg 95

Power losses in a cavity: Q. Jackson pg 371

\section[(ei fu)]{(era...)Leggi di Conservazione. Equilibrio statico.}
\begin{itemize*}
\item fusione di questa sezione?
\item Equazioni mechaniche nella sezione 1?
\item Equazioni conservazione e trasporto di energia parte 3?
\item \sout{Equazioni di base della struttura stellare}
\item Equazion of motion for spherical symmetry: $\tau_{ff}$, $\tau_{expl}$ ($\S 2.4$ kipp): la stella occupa stati di quasi equilibrio per gran parte della vita $\tau_{nucl}$
\item Kelvin-Helmholtz scale time ($\S 3.1-3.3$ kippen):
\item Equazioni par 4 cox (nella parte equilibrio struttura autogravitante): leggi di conservazione
\item Equazion of motion for spherical symmetry: $\tau_{ff}$, $\tau_{expl}$ ($\S 2.4$ kipp): la stella occupa stati di quasi equilibrio per gran parte della vita $\tau_{nucl}$
\item \sout{Equazioni struttura solare. Simmetria sferica: cosa trascuro.}
\begin{align*}
&\TDy{r}{p}=-\frac{Gm\rho}{r^2}&\intu{Momentum conservation along with Poisson's equation:}\\
&\TDy{r}{m}=4\pi r^2\rho\\
&\TDy{r}{T}=\nabla\frac{T}{p}\TDy{r}{p}\\
&\TDy{r}{L}=4\pi r^2[\rho\epsilon-\rho\TDof{t}\frac{u}{\rho}+\frac{p}{\rho}\TDy{t}{\rho}]
\end{align*}
\item Trascuro la rotazione e i compi magnetici.
\item \sout{The assumption $\ten{P}=IP$ where P is the thermodynamic pressure imply neglegible molecular and radiative viscosity, large-scale magnetic field and turbolence.}
\item \sout{temposcala dinamico:}
\item Connection between convective energy transport and local structure.
\item Convective instability: $\nabla_{Rad}>\nad{}=\Dcvar{\PDly{P}{T}}{Ad}$.
\item $\frac{1}{\kappa\rho}$ is the mean free path of photon.
\item Where energy is transported by radiation: $\nabla=\nabla_{Rad}=\frac{3}{16\pi acG}\frac{\kappa P}{T^4}\frac{l(r)}{m(r)}$.
\item onde propagazione frequenze plasma lunghezze caratteristiche frequenze di taglio (asymptotic description)
\item Plasma ideale: costante di accoppiamento
\item equazione di stato (stix pg 29) ???
\item The applicability of fluid approach (sh8u gas dynamics). mean free path and plasma frequency (sh8u gas dynamics)
\item Helium diffusion in the Sun
\begin{equation*}
    \PDy{t}{X}=R_H+\frac{1}{r^2\rho}\PDof{r}[r^2\rho(D_H\PDy{r}{X}+V_HX)]
\end{equation*}
$R_H$ Rate of change of H abbundance due to nuclear reactions, $D_H$ is the diffusion coefficient, $V_H$ is the settling speed.

\end{itemize*}

\stopcontents[chapters]

\end{document}