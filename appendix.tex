\documentclass[../main.tex]{subfiles}

\begin{document}

\chapter{Campio vettoriali solenoidali}
\PartialToc

\section{Helmholtz decomposition}

\subsection{poloidal–toroidal decomposition}

A poloidal–toroidal decomposition is a restricted form of the Helmholtz decomposition that is often used in the spherical-coordinates analysis of solenoidal vector fields, for example, magnetic fields and incompressible fluids. For a three-dimensional $\vec{F}=0$, such that $\scap{\nabla}{F}=0$ can be expressed as the sum of a toroidal and poloidal vector fields:

\begin{align*}
&\vec{F} =\vec{T} +\vec{P} =\\
&\nabla\wedge(\Psi(r,\theta,\phi)\hat{r} +\nabla\wedge(\nabla\wedge(\Phi\hat{r})
\end{equation*}
where $\hat{r}$  is a radial vector in spherical coordinates and $\vec{T}$ is a toroidal field

\begin{equation*}
\vec{T} =\nabla\wedge(\Psi\hat{r}
\end{equation*}
for scalar field $\Psi(r,\theta ,\phi )$ and where $\vec{P}$ is a poloidal field

\begin{equation*}
\vec{P} =\nabla\wedge\nabla\wedge(\Phi\hat{r} 
\end{equation*}

for scalar field $\Phi(r,\theta ,\phi )$.

This decomposition is symmetric in that the curl of a toroidal field is poloidal, and the curl of a poloidal field is toroidal. A toroidal vector field is tangential to spheres around the origin $\hat{r} \cdot \vec{T} =0$ while the curl of a poloidal field is tangential to those spheres $\hat{r}\cdot(\nabla\wedge\vec{P} )=0$.

The poloidal–toroidal decomposition is unique if it is required that the average of the scalar fields $\Psi$ and $\Phi$ vanishes on every sphere of radius $r$.

\stopcontents[chapters]

\end{document}