\documentclass[../main.tex]{subfiles}

\begin{document}


% the tikz-er2.sty package is available at:
% http://tagus.inesc-id.pt/~pcalado/tikzer2/tikz-er2.
\chapter{Da usare in tesi: rappels}
\PartialToc

\section{All tesi keywords}

\begingroup
\nocite{*}
\let\clearpage\relax
\printbibliography[filter=tesirappels,title={\textcolor{brilliantlavender}{Da mettere in SOS}}]
\endgroup


\chapter{Visual SSM}
\PartialToc

\section{SSM e HSSSM}

\begingroup %%% grafico tikz generale

\tikzstyle{every entity} = [fill=blue!30, draw=blue!50!black!100, drop shadow,text width=2cm]

\tikzstyle{every weak entity}=[fill=black!20!white!100]

\tikzstyle{every attribute} = [fill=yellow!100, draw=yellow, node distance=1cm, drop shadow,text width=1.5cm, fill opacity=0.5]

\tikzstyle{every relationship} = [top color=white, bottom color=red!20, draw=red!50!black!100, drop shadow,text width=2cm]

\centering
\scalebox{.25}{
\begin{tikzpicture}[node distance=1.5cm, every edge/.style={link}]

  \node[entity] (ssm) at (0,0) {SSM};
\draw[link,<-]  (ssm.120)--++(120:2cm) node[attribute,anchor=120+180] (hydro) {Equilibrio idrostatico};
\draw[link,<-]  (ssm.160)--++(160:2cm) node[attribute,anchor=160+180] (thermal) {Equilibrio termico locale};
\draw[link,<-]  (ssm.200)--++(200:2cm) node[attribute, anchor=200+180] (opacity) {Equilibrio radiativo + criterio di Schwartzchild};

\node[relationship] (calibration) [above =3cm of ssm] {Calibrazione luminosit\'a e raggio} edge[<-] (ssm);

\node[entity] (unphys) [below=2cm of ssm] {Incertezze fisica del modello} edge [<-] (ssm);
\draw[link,<->] (unphys.200)--++(200:4cm) node[attribute, anchor=200+180] (diffusion) {Diffusione};
\draw[link,<->] (unphys.215)--++(215:2cm) node[attribute, anchor=215+180] (MLT) {Modello convezione: MLT.};
\draw[link,<->] (unphys.250)--++(250:4cm) node[attribute, anchor=250+180] (modefreqsurf) {Influenza della zona superficiale sulle frequenze delle oscillazioni};

\node[entity] (surfaceobs) [above right=3cm and 3cm of ssm] {Osservabili stellari};
\draw[link,->] (surfaceobs.90)--++(90:2cm) node[attribute, anchor=90+180] {Misura metellicit\'a superficiale};

\node[entity] (modelobs) [below=3cm of surfaceobs.south east,anchor=north] {Discrepanze modello/osservazioni}; 
\node[entity,anchor=north west] (corrections) at (modelobs.south east) {Correzioni} edge [->,in=30,out=60] (ssm);
\node[entity,anchor=north west] (sismobs) at (surfaceobs.south east) {Osservabili sismologiche};
\draw[link,->]  (sismobs.45)--++(45:3cm) node[attribute, anchor=45+180] (freqs) {Frequenze oscillazioni adiabatiche};
\draw[link,->] (sismobs.5)--++(5:2cm) node[attribute, anchor=5+180] (radiuscz) {Raggio fondo zona convettiva};
\draw[link,->] (sismobs.8)--++(8:4cm) node[attribute, anchor=8+180] (heliumenv) {$Y_{ph}$};
\draw[link,->] (sismobs.345)--++(345:5cm) node[attribute, anchor=345+180] (soundspeed) {Velocit\'a del suono (isothermal): $c_s(r)/u(r)$};
\draw[link,->] (sismobs.30)--++(30:4cm) node[attribute,anchor=30+180] (densitybcz) {Densit\'a fondo zona convettiva $\rho_{cz}$};
\draw[link,->] (ssm.315)--++(315:2cm) node[entity, anchor=315+180] (ssmoscmodes) {Soluzione problema agli autovalori modi normali adiabatici};
\node[entity,below=1.5cm of modelobs] (inversion)  {Tecniche di inversione};
\draw[link,->] (sismobs.215)--(inversion.50);

\node[entity,anchor=north west] (neutrinoflux) at (sismobs.south east) {Flusso neutrini solari};

\draw[link,<->] (surfaceobs.200) -- (unphys.50) node [sloped,pos=0.5, anchor=north] (stellaracc) {accuratezza};
\draw[link,<->] (sismobs.270) -- (unphys.60) node [sloped,pos=0.5, anchor=south] (sismacc) {accuratezza};

\node[draw,purple,fit=(hydro)(thermal)(opacity)(unphys)(ssm),ellipse,label={[purple,label distance=1cm]120:Modello Solare standard}] (ssmapprox) at (hydro.west) {};
\node[weak entity,anchor=east] (ssmapproxlist) at (ssmapprox.west) {Oltre il SSM:
\begin{itemize*} \item Composizione superficiale \item modelli idrodinamici dell'atmosfera e inclusione di effetti NLTE \item Effetto screening degli elettroni sulle reazioni nucleari\end{itemize*} };

\end{tikzpicture}

}

\endgroup %% fine grafico tikz

\newrefcontext[sorting=hot]

{\let\clearpage\relax
\chapter{Modello solare}}
\PartialToc

\begin{refsection}[solarmodels.bib]

\section{SSM: input2output and uncertainties}
%\nocite{*}

\begingroup
\let\clearpage\relax

\printbibliography[filter=SSMinout,keyword={rev},title={\textcolor{bittersweet}{Biblio about ''Risultati SSM e loro incertezze''}}]

\printbibliography[filter=SSMinout, notkeyword={rev},title={\textcolor{bittersweet}{Other refs about ''Risultati SSM e loro incertezze''}}]

\endgroup

\end{refsection}

\begin{refsection}[KinHydRev.bib,StellarModelS.bib,RotationMagneticS.bib,sunmeasure.bib,solarmodels.bib]

%\nocite{*}
\printbibliography[title={\textcolor{Orchid}{Bibliografia modello solare e oltre}}]

\section{approssimazione equilibrio idrostatico}

\section{Campo magnetico, rotazione, trasporto momento angolare}


\begingroup
\let\clearpage\relax

\printbibliography[filter=staticdeviation,keyword={rev},title={\textcolor{violetw}{Biblio about ''Deviazioni da equlibrio idrostatico: campo magnetico, rotazione, trasporto momento angolare, turbolenza''}}]

\printbibliography[filter=staticdeviation, notkeyword={rev},title={\textcolor{violetw}{Other refs about ''Deviazioni da equlibrio idrostatico: campo magnetico, rotazione, trasporto momento angolare, turbolenza''}}]

\endgroup

\end{refsection}


\begin{refsection}[StellarModelS.bib,starRev.bib,cox.bib]

%\nocite{*}

\section{Energia interna ed equilibrio termico}

\subsection{Equazione di stato}

\begingroup
\let\clearpage\relax
\printbibliography[filter=eos,keyword={rev},title={\textcolor{atomictangerine}{Biblio about ''Equazione di stato''}}]

\printbibliography[filter=eos,notkeyword={rev},title={\textcolor{atomictangerine}{Other refs about ''Equazione di stato''}}]

\endgroup

\subsection{Energia interna}

Da cui ricavo le grandezze termodinamiche tipo $c_V$ etc


\subsection{Trasporto radiativo}

\begingroup
\let\clearpage\relax

\printbibliography[filter=radiativetransport,keyword={rev},title={\textcolor{darkgray}{Biblio about  ''Trasporto radiativo''}}]

\printbibliography[filter=radiativetransport,notkeyword={rev},title={\textcolor{darkgray}{Other refs about ''Trasporto radiativo''}}]

\endgroup

\subsection{Trasporto convettivo}

\begingroup
\let\clearpage\relax

\printbibliography[filter=convection,keyword={rev},title={\textcolor{brightcerulean}{Biblio about ''Convective envelope''}}]

\printbibliography[filter=convection,notkeyword={rev},title={\textcolor{brightcerulean}{Other refs about ''Convective envelope''}}]

\endgroup

Modelli convezione ???

\subsection{Reazioni nucleari}

\begingroup
\let\clearpage\relax

\printbibliography[filter=nuclear,keyword={rev},title={\textcolor{cadmiumorange}{Biblio about ''Reazioni nucleari e neutrini''}}]

\printbibliography[filter=nuclear,notkeyword={rev},title={\textcolor{cadmiumorange}{Other refs about ''Reazioni nucleari e neutrini''}}]

\endgroup

\subsection{Fattore astrofisico}

\subsection{Neutrini}

\end{refsection}


\begin{refsection}[StellarModelS.bib,solarmodels.bib,KinHydRev.bib]

%\nocite{*}
\begingroup
\let\clearpage\relax

\printbibliography[filter=stellarplasma,keyword={rev},title={\textcolor{lemon}{Biblio about ''Fisica del (trasporto nel) plasma stellare''}}]

\printbibliography[filter=stellarplasma,notkeyword={rev},title={\textcolor{lemon}{Other refs about ''Fisica del (trasporto nel) plasma stellare''}}]

\endgroup

\section{Non uniform state for gas mixture}



\subsection{Boltzmann's equation}

\begin{align*}
&\PDy{t}{f_1}+\vec{c}_1\cdot\PDy{\vec{r}_1}{f_1}+\vec{F}_1\cdot\PDy{\vec{c}_1}{f_1}=\left(\PDy{t}{f_1}\right)_c\\
&\PDy{t}{f_2}+\vec{c}_2\cdot\PDy{\vec{r}_2}{f_2}+\vec{F}_2\cdot\PDy{\vec{c}_2}{f_2}=\left(\PDy{t}{f_2}\right)_c&\intertext{dove:}\\
&
\end{align*}

\subsection{Entropy}

For a gas in uniform steady state

\begin{align*}
&H=\int f\ln{f}\,d^3c=n\overline{\ln{f}}\\
&=n[\ln{n}+\frac{3}{2}\ln{\frac{m}{2\pi kT}}-\frac{3}{2}]&\intertext{Con $M$ total mass of the gas and $\midfrac{M}{\rho}=\midfrac{M}{nm}$}\\
&H_0=\int H\,d^r=\frac{M}{m}[\ln{n}+\frac{3}{2}\ln{\frac{m}{2\pi kT}}-\frac{3}{2}]\\
&(pg80-81)\\
&S=-kH_0
\end{align*}

\subsection{Definition/Notation: binary encounters}

$\vec{c}_1,\vec{c}_2$ denote velocity, and $\vec{c}_1',\vec{c}_2'$ denotes velocities of two molecules after encounter

\begin{align*}
&m_0=m_1+m_2,M_1=\frac{m_1}{m_0},M_2=\frac{m_2}{m_0}&\intertext{the center of mass of the two molecules will move uniformly throughout the encounter: its constant velocity is give by}\\
&m_0\vec{G}=m_1\vec{c}_1+m_2\vec{c}_2=m_1\vec{c}_1'+m_2\vec{c}_2'&\intertext{the relative velocity of second resp. to first and viceversa,before and after encounter, are}\\
&\vec{g}_{21}=\vec{c}_2-\vec{c}_1=-\vec{g}_{12},\ \vec{g}_{21}'=\vec{c}_2'-\vec{c}_1'=-\vec{g}_{12}'&\intertext{thus}\\
&\vec{c}_1=\vec{G}+M_2\vec{g}_{12}, \vec{c}_2=\vec{G}+M_1\vec{g}_{21}\\
&\vec{c}_1'=\vec{G}+M_2\vec{g}_{12}', \vec{c}_2'=\vec{G}+M_1\vec{g}_{21}'
\end{align*}

Let $\vec{k}$ be a unit vector in the direction connecting the center of mass of second molecule at closest approach to the first in a system in which the last one is at rest
\begin{align*}
&\vec{g}_{21}-\vec{g}_{21}'=2(\vec{g}_{21}\cdot\vec{k})\vec{k}=-2(\vec{g}_{21}'\cdot\vec{k})\vec{k}\\
&\vec{c}_1'-\vec{c}_1=2M_2(\vec{g}_{21}\cdot\vec{k})\vec{k}=-2M_2(\vec{g}_{21}'\cdot\vec{k})\vec{k}\\
&\vec{c}_2'-\vec{c}_2=-2M_1(\vec{g}_{21}\cdot\vec{k})\vec{k}=2M_1(\vec{g}_{21}'\cdot\vec{k})\vec{k}
\end{align*}

\subsection{Definition/Notation: Integrals}

Sia $F(\vec{c})$
\begin{align*}
&n_1^2I_1(F)=\iint f_1^{(0)}f^{(0)}(F_1+F-F_1'-F')\,d^3k\,d^3c\\
&[F,G]_1=\int G_1I_1(F)\,d^3c_1\\
&[F_1+G_2,H_1+K_2]_{12}=\int F_1I_{12}(H_1+K_2)\,d^3c_1+\int G_2I_{21}(H_1+K_2)\,d^3c_2\\
&n_1n_2\{F,G\}=n_1^2[F,G]_1+n_1n_2[F_1+F_2,G_1+G_2]_{12}+n_2^2[F,G]_2
\end{align*}


\subsection{Changes of molecular properties}

\begin{align*}
&\MDy{(n_1\overline{\phi}_1)}+n_1\overline{\phi}_1\PDof{\vec{r}}\cdot\vec{c}_0\\
&-n_1\left\{ \MDy{\overline{\phi}_1}\overline{\vec{C}_1\cdot\PDy{\vec{r}}{\phi_1}}+(\vec{F}_1-\MDy{\vec{c}_0})\cdot\overline{\PDy{\vec{C}_1}{\phi_1}}\right.\\
&\left.-\overline{\PDy{\vec{C}_1}{\phi_1}\vec{C}_1}:\PDof{\vec{r}}\vec{c}_0\right\}=n_1\Delta\overline{\phi_1}
\end{align*}


\subsection{Diffusion and thermal diffusion}
(\cite[pp. 143-144]{chapman52gas})

\begin{align*}
&\overline{\vec{c}_1}-\overline{\vec{c}_2}=\overline{\vec{C}_1}-\overline{\vec{C}_2}=\frac{1}{n_1}\int f_1C_1\,d^3c_1-\frac{1}{n_2}\int f_2C_2\,d^3c_2\\
&=-\frac{1}{3}n_1n_2[\{\vec{D},\vec{D}\}n\vec{d}_{12}+\{\vec{D},\vec{A}\}\PDy{\vec{r}}{\ln{T}}]
\end{align*}


\end{refsection}



\begin{refsection}[solarmodels.bib,innocenti.bib,sunmeasure.bib]

\section{Incertezze nel SSM}

\begingroup
\nocite{*}
\let\clearpage\relax

%276
\printbibliography[filter=ssmunphys,keyword={rev},title={\textcolor{ochre}{Biblio about: ''Incertezze nei parametri e nella fisica usata nel SSM''}}]
\printbibliography[filter=ssmunphys,notkeyword={rev},title={\textcolor{ochre}{Other refs about: ''Incertezze nei parametri e nella fisica usata nel SSM''}}]
\endgroup

\subsection{Calibrazione modello solare}

incertezza su $S_{11}$
incertezza composizione
\begin{itemize*}
\item pressione gas
\item pressione turbolenta == kinetic pressure
\item radiation pressure $-->$ photon absorption
\item magnetic pressure
\end{itemize*}
tabella input del mss: eta, luminosita, raggio,
Incertezze negli input del SSM e quindi su $\alpha$ e $Y$

\subsection{Incertezze nelle quantit\'a teoriche del modello solare rilevanti eliosismologicamente dovute alle incertezze negli input/parametri}

%\renewcommand{\footrefs}{\citeauthor*{bahcall200610}\citeyear{bahcall200610} vs \citeauthor*{boothroyd2003our}\citeyear{boothroyd2003our}.}

\end{refsection}


{\let\clearpage\relax
\chapter{Oscillazioni solari: forward problem. Eccitazione.}}
\PartialToc

\begin{refsection}[HAseismRev.bib,cox.bib,KinHydRev.bib,SModes.bib,VstarsRev.bib,waves.bib]

\section{Eccitazione/Damping dei modi vs instabilit\'a}

\subsection{Turbolenza: eccitazione stocastica e viscosit\'a}
\subsection{tempo di vita: damping}
\subsection{Effetti non lineari}
\subsection{Overstability vs stochastic excitation}

\section{Comportamento asintotico dei modi}

\end{refsection}


{\let\clearpage\relax
\chapter{Oscillazioni solari: osservazione. Rimozione degenerazione: m frequency splitting.}}
\PartialToc

\begin{comment}
\begin{refsection}[RotationMagneticS.bib,SModes.bib,ObsModesTech.bib]

\begingroup
%\nocite{*}
\let\clearpage\relax
\traceon
\printbibliography[check=modesfreqobs,title={\textcolor{ochre}{Biblio about: ''Osservazione dei modi: strumenti e splitting dovuto alla rotazione.''}}]
\printbibliography[check=modesfreqobs,notkeyword={rev},title={\textcolor{ochre}{Other refs about: ''Osservazione dei modi: strumenti e splitting dovuto alla rotazione.''}}]
\traceoff
\endgroup

\section{Problematiche osservative}

Na $D$ lines: $\lambda=\SI{589.6}{\nano\meter},\SI{589.0}{\nano\meter}$; SOHO (MDI): Ni \SI{678.8}{\nano\meter}; K Fraunhofer line \SI{770}{\nano\meter}; Ca \SI{643.9}{\nano\meter}; K \SI{769.9}{\nano\meter}

\subsection{Doppler measurement: resonant scattering cell (GOLF)}

%scatteringcell
Principio funzionamento spettroscopio scatering risonante. Da \cite{brookes1978resonant}.

\subsection{separazione contributi modi}
\subsection{Frequency precision}

\end{refsection}
\end{comment}

{\let\clearpage\relax
\chapter{Inversione rotazione.}}
\PartialToc

\begin{refsection}[SModes.bib,RotationMagneticS.bib,VstarsRev.bib,HelioInversion.bib]

\section{First order degenerate perturbation: ROTATION.}

\renewcommand{\footrefs}{Refs: \cite{ritzwoller1991unified}}

\subsection{Approccio perturbativo: Degenerate perturbation}

Per descrivere lo splitting in m generato dalla rotazione considero al primo ordine i modi del multipletto (n,l) dato che la distanza tra modi di diverso n o l \'e molto maggiore della separazione per m differenti nello stesso multipletto.

La soluzione del problema diretto (determinazione degli splitting in m) che sar\'a trovato

\subsection{Scomposizione tradizionale}

\begin{align*}
&\Omega(r,\theta)=\sum_{k=0,2,4,\ldots}\Omega_k(r)P_k(\cos{\theta})\\
&\omega_{nl}^m=\omega_{nl}+l\sum_{i=1}^M\tensor[_n]{a}{_l_i}P_i(\midfrac{m}{l})
\end{align*}

\subsection{Forward problem: uso delle armoniche sferiche vettoriali}

Decomposizione di laminar velocity field into poiloidal and toroidal component $\vec{P}$ and $\vec{T}$
\begin{align*}
&\vec{v}_{rot}=\vec{\Omega}(r,\theta,\phi)\wedge\vec{r}\\
&=\sumzi{s}\sumft{t=-s}{s}[\vec{P}_t^s(r,\theta,\phi)+T_s^t(r,\theta,\phi)]
\end{align*}

Le funzioni $P$ e $T$ sono caratterizzate dai coefficienti dipendenti da r dell'espansione in armoniche sferiche $u_s^t(r)$, $v_s^t(r)$, $w_s^t(r)$:
\begin{align*}
&\vec{P}_t^s(r,\theta,\phi)=u_s^t(r)Y_s^t(\theta,\phi)\hat{r}+v_s^t(r)\nabla_1Y_s^t(\theta,\phi)\\
&\vec{T}_t^s(r,\theta,\phi)=-w_s^t(r)\hat{r}\wedge\nabla_1Y_s^t(\theta,\phi)\\
&\nabla_1=r[\nabla-\hat{r}(\hat{r}\cdot\nabla)]=\hat{\theta}\PDof{\theta}+\frac{\hat{\phi}}{\sin{\theta}}\PDof{\phi}\\
&Y_s^t(\theta,\phi)=(-1)^t[\frac{2s+1}{4\pi}\frac{(t-s)!}{(t+s)!}]\expy{\midfrac{1}{2}}P_s^t(\cos{\theta}
)\exp{it\phi}
\end{align*}

\begin{definition}{Funzioni armoniche sferiche vettoriali}

\begin{align*}
&ttt
\end{align*}

\end{definition}

\renewcommand{\footrefs}{{}}

\subsection{Misurazione spitting dei modi dovuto alla rotazione e campi magnetici interni}


\end{refsection}


{\let\clearpage\relax
\chapter{Oscillazioni solari: inversione. Tecniche di inversione e fit. HSSM. Oltre il SSM.}}
\PartialToc

\begin{refsection}[HelioInversion.bib,beyond.bib]


\section{Tecniche numeriche}


\end{refsection}


\end{document}