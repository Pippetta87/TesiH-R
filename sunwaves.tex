\chapter{Onde magneto-idrodinamiche}
\PartialToc

\section{Cenni storici su MHD waves}
\cite{alfvenweves}

\begin{itemize}
\item 1942: Alfv\'en suggests the existence of electromagnetic-hydromagnetic waves in a paper published in Nature.
\item 1949: Laboratory experiments by S. Lundquist produce such waves in magnetized mercury, with a velocity that approximated Alfvén's formula.
\item 1949: Enrico Fermi uses Alfv\'en waves in his theory of cosmic rays. According to Alex Dessler in a 1970 Science journal article, Fermi had heard a lecture at the University of Chicago, Fermi nodded his head exclaiming ''of course'' and the next day, the physics world said ''of course''.
\item 1950: Alfv\'en publishes the first edition of his book, Cosmical Electrodynamics, detailing hydromagnetic waves, and discussing their application to both laboratory and space plasmas.
\item 1952: Additional confirmation appears in experiments by Winston Bostick and Morton Levine with ionized helium
\item 1954: Bo Lehnert produces Alfv\'en waves in liquid sodium
\item 1958: Eugene Parker suggests hydromagnetic waves in the interstellar medium
\item 1958: Berthold, Harris, and Hope detect Alfv\'en waves in the ionosphere after the Argus nuclear test, generated by the explosion, and traveling at speeds predicted by Alfv\'en formula.
\item 1958: Eugene Parker suggests hydromagnetic waves in the Solar corona extending into the Solar wind.
\item 1959: D. F. Jephcott produces Alfv\'en waves in a gas discharge
\item 1959: C. H. Kelley and J. Yenser produce Alfv\'en waves in the ambient atmosphere.
\item 1960: Coleman, et al., report the measurement of Alfv\'en waves by the magnetometer aboard the Pioneer and Explorer satellites
\item 1960: Sugiura suggests evidence of hydromagnetic waves in the Earth's magnetic field
\item 1961: Normal Alfv\'en modes and resonances in liquid sodium are studied by Jameson
\item 1966: R.O.Motz generates and observes Alfv\'en waves in mercury
\item 1970 Hannes Alfv\'en wins the 1970 Nobel Prize in physics for ''fundamental work and discoveries in magneto-hydrodynamics with fruitful applications in different parts of plasma physics''
\item 1973: Eugene Parker suggests hydromagnetic waves in the intergalactic medium
\item 1974: Hollweg suggests the existence of hydromagnetic waves in interplanetary space
\item 1974: Ip and Mendis suggests the existence of hydromagnetic waves in the coma of Comet Kohoutek.
\item 1984: Roberts et al. predict the presence of standing MHD waves in the solar corona, thus leading to the field of coronal seismology.
\item 1999: Aschwanden, et al. and Nakariakov, et al. report the detection of damped transverse oscillations of solar coronal loops observed with the EUV imager on board the Transition Region And Coronal Explorer (TRACE), interpreted as standing kink (or ''Alfv\'enic'') oscillations of the loops. This fulfilled the prediction of Roberts et al. (1984).
\item 2007: Tomczyk, et al., report the detection of Alfv\'enic waves in images of the solar corona with the Coronal Multi-Channel Polarimeter (CoMP) instrument at the National Solar Observatory, New Mexico. These waves were interpreted as propagating kink waves by Van Doorsselaere et al. (2008)
\item 2007: Alfv\'en wave discoveries appear in articles by Jonathan Cirtain and colleagues, Takenori J. Okamoto and colleagues, and Bart De Pontieu and colleagues. De Pontieu's team proposed that the energy associated with the waves is sufficient to heat the corona and accelerate the solar wind. These results appear in a special collection of 10 articles, by scientists in Japan, Europe and the United States, in the 7 December issue of the journal Science. It was demonstrated that those waves should be interpreted in terms of kink waves of coronal plasma structures by Van Doorsselaere, et al. (2008); Ofman and Wang (2008); and Vasheghani Farahani, et al. (2009).
\item 2008: Kaghashvili et al. proposed how the detected oscillations can be used to deduct properties of Alfven waves. The mechanism is based on the formalism developed by the Kaghashvili and his collaborators.[9]
\item 2011: Experimental evidence of Alfvén wave propagation in a Gallium alloy.
\end{itemize}


\section{Types of magnetohydrodynamic waves}
\cite{coronalseism}


There are several distinct kinds of MHD modes which have quite different dispersive, polarisation, and propagation properties:

\begin{itemize}
\item Kink (or transverse) modes, which are oblique fast magnetoacoustic (also known as magnetosonic waves) guided by the plasma structure; the mode causes the displacement of the axis of the plasma structure. These modes are weakly compressible, but could nevertheless be observed with imaging instruments as periodic standing or propagating displacements of coronal structures, e.g. coronal loops. The frequency of transverse or "kink" modes is given by the following expression:

\begin{equation*}
\omega_{K}=\sqrt{\frac{2k_{z}B^{2}}{\mu (\rho_{i}+\rho_{e})}}
\end{equation*}

For kink modes the parameter the azimuthal wave number in a cylindrical model of a loop, m is equal to 1, meaning that the cylinder is swaying with fixed ends.

\item Sausage modes, which are also oblique fast magnetoacoustic waves guided by the plasma structure; the mode causes expansions and contractions of the plasma structure, but does not displace its axis. These modes are compressible and cause significant variation of the absolute value of the magnetic field in the oscillating structure. The frequency of sausage modes is given by the following expression:
\begin{equation*}
\omega_{S}=\sqrt{\frac{k_{z}^{2}B^{2}}{\mu \rho_{e}}}
\end{equation*}

For sausage modes the parameter m is equal to 0; this would be interpreted as a "breathing" in and out, again with fixed endpoints.

\item Longitudinal (or slow, or acoustic) modes, which are slow magnetoacoustic waves propagating mainly along the magnetic field in the plasma structure; these mode are essentially compressible. The magnetic field perturbation in these modes is negligible. The frequency of slow modes is given by the following expression:

\begin{equation*}
\omega_L=\sqrt{k_z^2\frac{C_s^2C_A^2}{C_s^2+C_A^2}}
\end{equation*}

Where we define $C_s$ as the sound speed and $C_{A}$ as the Alfvén velocity.

\item Torsional (Alfv\'en or twist) modes are incompressible transverse perturbations of the magnetic field along certain individual magnetic surfaces. In contrast with kink modes, torsional modes cannot be observed with imaging instruments, as they do not cause the displacement of either the structure axis or its boundary.

\begin{equation*}
\omega_{A}=\sqrt{\frac{k_{z}^{2}B^{2}}{\mu \rho _{i}}}
\end{equation*}

\end{itemize}

\chapter{MHD waves in the Sun}
\PartialToc

\section{Granulation, MHD waves and the solar corona}
\cite{alf47granulation}


\stopcontents[chapters]
