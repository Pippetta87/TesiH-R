\documentclass[../main.tex]{subfiles}
\begin{document}
\tableofcontents

%%%%%%%%%%%%%%%%%%%%%%%%%%%%%%%%%%%%
\pgfkeys{/pgf/fpu=true}
%%%%%%%%%%%%%%%%%%%%%%%%%%%%%%%%%%%% Define Constant for pgf

% G
%https://en.wikipedia.org/wiki/Gravitational_constant
\pgfmathparse{6.674*(10^(-11))} %m3 Kg-1 s-2
\edef\Gconst{\pgfmathresult}
\pgfmathparse{6.67259*(10^(-8))} % cm3 g-1 s-2 relative error 128 ppm
\edef\Gcgs{\pgfmathresult}

%%Boltzmann constant
\pgfmathparse{1.3805*(10^(-16))} %1.3805e-16 erg/degC
\edef\kerg{\pgfmathresult}
\pgfmathparse{8.617*(10^(-5))} %8.617e-5 eV/Ke
\edef\kelettronvolt{\pgfmathresult}
\pgfmathparse{2.0836618*(10^(10))} %2.0836618e10 Hz/Kelvin
\edef\khertz{\pgfmathresult}

%%% speed of light
\pgfmathparse{299792458*10^2} %cm/second
\edef\light{\pgfmathresult}


%electron
\pgfmathparse{0.510998910}  %Mev
\edef\electronm{\pgfmathresult}

%% massa
\pgfmathparse{1989*10^(27)} %Kg
\edef\massaS{\pgfmathresult}


%%Standard_gravitational_parameter
%https://en.wikipedia.org/wiki/Standard_gravitational_parameter
%http://ssd.jpl.nasa.gov/?constants 

\pgfmathparse{132712*10^(15)}   % m3 s-2 % standard gravitational parameter
\edef\standardpar{\pgfmathresult}


%% Radius
%http://nssdc.gsfc.nasa.gov/planetary/factsheet/sunfact.html
\pgfmathparse{696000*10^3} % m
\edef\radius{\pgfmathresult}

%% densita media
\edef\densitym{1408}% mean density Kg/m^3
\pgfmathparse{\densitym*10^(-3)} % m
\edef\densitymcgs{\pgfmathresult}% mean density g/cm^3

%% Accelerazione di gravità in superficie
\pgfmathparse{(274)}
\edef\gravitysurface{\pgfmathresult}    % gravity acceleration at sun's surface in m/s^2

%% central temperature

\pgfmathparse{1.571*10^7} %kelvin
\edef\coreT{\pgfmathresult}

%% central pressure
\pgfmathparse{2.477*10^(11)*10^5} %Pressure bar to N/m^2
\edef\pressurec{\pgfmathresult}
%% 1 bar = 10^5 N/m^2


%% Solar composition
\pgfmathparse{0.70}
\edef\X{\pgfmathresult}

%% Solar irradiance

%%Solar luminosity
%http://www.astro.wisc.edu/~dolan/constants.html
\pgfmathparse{3.846*10^(33)}    %%erg*s-1
\edef\luminosityS{\pgfmathresult}

%%% Reazioni nucleari
\pgfmathparse{6.3*10^(18)}  %% erg g^-1 %
\edef\QHHe{\pgfmathresult}


\chapter{Tempi caratteristici}

\section{Tempo-scala evoluzione dinamica}

\subsection{Tempo collasso}
\pgfmathparse{sqrt((\radius)/(\massaS*\Gconst/(\radius)^2))/60}

\edef\tff{\pgfmathresult}

$\tau_{ff}\approx$\pgfmathprintnumber{\tff} min 

\subsection{Tempo esplosione}
\pgfmathparse{\radius*sqrt(\densitym/(\Gconst*\massaS*\densitym/\radius))/60}
\edef\tesp{\pgfmathresult}

$\tau_{exp}\approx$ \pgfmathprintnumber{\tesp} ??

\subsection{Tempo idro-dinamico}
\pgfmathparse{sqrt((\radius)^3/\standardpar)/60}
\edef\thydro{\pgfmathresult}

\pgfmathparse{1/2*1/sqrt(\Gconst*\densitym)/60} 
\pgfmathprintnumber{\pgfmathresult} min
\pgfmathprintnumber{\thydro} min

\section{Tempo collasso gravitazionale}

\pgfmathparse{(\Gcgs*(\massaS*1000)^2)/(2*(100*\radius)*\luminosityS)/(60*60*24*365)} yr
$\tkh{}=$\pgfmathprintnumber{\pgfmathresult}


\section{Tempo nucleare}

\pgfmathparse{15/100}\edef\Hef{\pgfmathresult}

\pgfmathparse{(\massaS*1000*\Hef*0.75*\QHHe/\luminosityS)/(60*60*24*365)}\edef\taun{\pgfmathresult}

Stima energia nucleare disponibile per fusione di tutto idrogeno in He:\pgfmathprintnumber{\taun} 

\section{Periodo fondamentale}

\subsection{Mean sound speed}
\pgfmathparse{5/3} 
\edef\Guno{\pgfmathresult}% Gammka 1 per gas completamente ionizzato

\pgfmathparse{2.477*10^(11)} 
\edef\pressureC{\pgfmathresult}%Central pressure:     2.477 x 1011 bar

\pgfmathparse{1.571*10^7} 
\edef\temperatureC{\pgfmathresult}%Central temperature:  1.571 x 107 K

\edef\q{1.5}  %q=0.6 constant density, q=1.5 MS stars, q=3/(5-n) for polytropic of index n
\pgfmathparse{\q*\Gcgs*((\massaS*10^3)^2)/(\radius*10^2)}
\edef\potentialEG{\pgfmathresult}% Energia potenziale gravitazionale: valore assoluto
 Energia potenziale gravitazionele ($||$):\pgfmathprintnumber{\potentialEG}.


\pgfmathparse{((\potentialEG*\Guno)/(\massaS*3*(10^3)))/((10^5)^2)}
\edef\soundsqrM{\pgfmathresult}%  squared sound speed 
Velocit\'a del suono media quadra: \pgfmathprintnumber{\soundsqrM} \si{\square\kilo\meter\per\square\second}.

\pgfmathparse{sqrt(0.1*10^6)}\edef\soundsMD{\pgfmathresult}
Velocit\'a del suono media (Dalsgaard 85): \pgfmathprintnumber{\soundsMD} \si{\kilo\meter\per\second}.

\pgfmathparse{sqrt(\soundsqrM)}\edef\soundsM{\pgfmathresult}
Velocit\'a del suono media (viriale): \pgfmathprintnumber{\soundsM} \si{\kilo\meter\per\second}.


\pgfmathparse{2*pi*(sqrt(\soundsqrM*10^(10)))/(2*\radius*10^2)}
\edef\omegafond{\pgfmathresult}%  frequenza modo fondamentale 
\pgfmathprintnumber{\omegafond}

\pgfmathparse{2*pi/\omegafond/3600}
\edef\pifond{\pgfmathresult}%  periodo modo fondamentale 
Periodo fondamentale: \pgfmathprintnumber{\pifond}.

\pgfmathparse{0.04*24}
\edef\qgrande{\pgfmathresult}%  pulsation constant Q
Pulsation constant: \pgfmathprintnumber{\qgrande}.

\chapter{Deubner75}

\pgfmathparse{0.5*10^(-6)*\radius}
\edef\ldeubner{\pgfmathresult}%  l tipico
l tipico osservazioni deubner: \pgfmathprintnumber{\ldeubner}.


\chapter{Equazione di stato}

\section{Esponente adiabatico}

\pgfmathparse{(3/2)*(\kelettronvolt)*\coreT}% Energia cinetica media core
\edef\avrkin{\pgfmathresult}%
Energia cinetic media core: \pgfmathprintnumber{\avrkin}\si{\electronvolt}.

\pgfmathparse{-(((2+2*\X)/(3+5*\X))*\coreT)/((\electronm*10^6)/(\kelettronvolt))}
\edef\dgammarel{\pgfmathresult}%
$\frac{\Lvar{\Gamma_1}}{\Gamma_1}=\pgfmathprintnumber{\dgammarel}$.


\chapter{Caratteristiche delle oscillazioni}

\section{Profondit\'a cavit\'a acustica}

%\pgfmathparse{10^(-3)}% Pulsazìione tipo secondi^-1
\pgfmathparse{((5000)*10^(-6)*(2*pi)}% Pulsazìione tipo microhertz
\edef\pulsationalone{\pgfmathresult}%

\pgfmathparse{0.1*10^(-6)} %Numero d'onda orizzontale tipo Metri^-1
\edef\wnumberone{\pgfmathresult}%

\pgfmathparse{(\pulsationalone)^2/(((\wnumberone)^2)*(5/3-1)*\gravitysurface)}

\edef\depthacustic{\pgfmathresult}%
La profondit\'a one $\delta=\pgfmathprintnumber{\depthacustic}\si{\meter}$ quindi $\frac{\delta}{\rsun{}}=\pgfmathparse{\depthacustic/\radius}\pgfmathprintnumber{\pgfmathresult}$, per un $L\approx\pgfmathparse{\radius*\wnumberone}\pgfmathprintnumber{\pgfmathresult}$

\chapter{Modello solare}

\section{Parametri fondamentali}

\pgfmathparse{(\luminosityS/\light^2)/\massaS}
\pgfmathprintnumber{\pgfmathresult}

\pgfkeys{/pgf/fpu=false}

\end{document}
